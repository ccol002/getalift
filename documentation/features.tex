\part{Features}

\chapter{Backend API}

\section{Ready}

Here is a list of the routes already availables from the backend API. All the routes below are preceded by \url{http://ip_server:port/api}

\subsection{Global}
\begin{itemize}
	\item GET "/"
	\begin{itemize}
		\item Returns "Hello World"
	\end{itemize}
\end{itemize}

\subsection{Users}
\begin{itemize}
	\item POST "/users"
	\begin{itemize}
		\item Create a user
	\end{itemize}
\end{itemize}

\begin{itemize}
	\item POST "/auth"
	\begin{itemize}
		\item Authenticate a user
	\end{itemize}
\end{itemize}

\begin{itemize}
	\item GET "/users"
	\begin{itemize}
		\item This route send back every public informations about every users. It can be a bit heavy with a lot of users.
	\end{itemize}
\end{itemize}

\begin{itemize}
	\item GET "/users/:usrid"
	\begin{itemize}
		\item This route send back the public informations about the chosen user
	\end{itemize}
\end{itemize}


\begin{itemize}
	\item PUT "/users/:usrid"
	\begin{itemize}
		\item This route update the information about the chosen user.
	\end{itemize}
\end{itemize}


\begin{itemize}
	\item DELETE "/users/:usrid"
	\begin{itemize}
		\item This route deletes the chosen user.
	\end{itemize}
\end{itemize}

\subsection{Routes}
\begin{itemize}
	\item GET "/routes"
	\begin{itemize}
		\item This route send back every informations about all the routes. It can be a bit heavy with a lot of routes.
	\end{itemize}
\end{itemize}


\begin{itemize}
	\item GET "/routes/:routeid"
	\begin{itemize}
		\item This route send back the public informations about the chosen route.
	\end{itemize}
\end{itemize}


\begin{itemize}
	\item GET "driverroutes/:driverid"
	\begin{itemize}
		\item This route send back the public informations about all the routes from a specific driver.
	\end{itemize}
\end{itemize}

\begin{itemize}
	\item PUT "/routes"
	\begin{itemize}
		\item This route create a new Route in the database. It search the optimal directions with the google maps API, in order to store the best route.
	\end{itemize}
\end{itemize}

\begin{itemize}
	\item DELETE "/routes/:routeid"
	\begin{itemize}
		\item This route deletes the chosen route.
	\end{itemize}
\end{itemize}

\begin{itemize}
	\item POST "/routes/findTarget"
	\begin{itemize}
		\item 	This route can be used in order to search for a route that match specific parameters.
	\end{itemize}
\end{itemize}

\subsection{Route Dates}
\begin{itemize}
	\item GET "/routedate/:routeid"
	\begin{itemize}
		\item This route send back the public informations about the chosen routedate.
	\end{itemize}
\end{itemize}

\begin{itemize}
	\item GET "/routedate"
	\begin{itemize}
		\item This route returns all the routedates for all the routes.
	\end{itemize}
\end{itemize}

\subsection{Rides}
\begin{itemize}
	\item GET "/rides"
	\begin{itemize}
		\item This route send back every informations about all the rides. It can be a bit heavy with a lot of rides.
	\end{itemize}
\end{itemize}

\begin{itemize}
	\item GET "/rides/:rideid"
	\begin{itemize}
		\item This route send back the public informations about the chosen ride.
	\end{itemize}
\end{itemize}

\begin{itemize}
	\item POST "/rides"
	\begin{itemize}
		\item This route create a ride in the database.
	\end{itemize}
\end{itemize}

\begin{itemize}
	\item PUT "/rides/:rideid"
	\begin{itemize}
		\item This route update the information about the chosen ride.
	\end{itemize}
\end{itemize}

\begin{itemize}
	\item  DELETE "/rides/:rideid"
	\begin{itemize}
		\item This route deletes the chosen ride.
	\end{itemize}
\end{itemize}

\subsection{Passenger}
\begin{itemize}
	\item GET "/passengers"
	\begin{itemize}
		\item This route send back every informations about all the passengers. It can be a bit heavy with a lot of passengers.
	\end{itemize}
\end{itemize}

\begin{itemize}
	\item GET "/passengers/:passid"
	\begin{itemize}
		\item This route send back the public informations about the chosen passenger.

	\end{itemize}
\end{itemize}

\begin{itemize}
	\item POST "/passenger"
	\begin{itemize}
		\item 	This route can create a passenger in the database.
	\end{itemize}
\end{itemize}

\begin{itemize}
	\item PUT "/passenger/:passid"
	\begin{itemize}
		\item This route update the information about the chosen passenger.
	\end{itemize}
\end{itemize}

\begin{itemize}
	\item DELTE "/passenger/:passid"
	\begin{itemize}
		\item This route deletes the chosen passenger.
	\end{itemize}
\end{itemize}

\subsection{Ratings}
\begin{itemize}
	\item GET "/ratings"
	\begin{itemize}
		\item 	This route send back every informations about all the ratings. It can be a bit heavy with a lot of rates.
	\end{itemize}
\end{itemize}

\begin{itemize}
	\item GET "/ratings/:rateid"
	\begin{itemize}
		\item This route send back the public informations about the chosen rate.
	\end{itemize}
\end{itemize}

\begin{itemize}
	\item POST "/ratings"
	\begin{itemize}
		\item This route create a rate in the database.
	\end{itemize}
\end{itemize}

\begin{itemize}
	\item 
	\begin{itemize}
		\item 
	\end{itemize}
\end{itemize}

\begin{itemize}
	\item DELETE "/ratings/:rateid"
	\begin{itemize}
		\item This route update the information about the chosen rating.
	\end{itemize}
\end{itemize}

\subsection{Favorite Route}
\begin{itemize}
	\item GET "/favoriteRoute/:userId"
	\begin{itemize}
		\item This routes returns all the favorites routes of a chosen user.
	\end{itemize}
\end{itemize}

\begin{itemize}
	\item POST "/favoriteRoute"
	\begin{itemize}
		\item This route create a favorite route.
	\end{itemize}
\end{itemize}

\begin{itemize}
	\item DELETE "/favoriteRoute"
	\begin{itemize}
		\item This route delete a favorite route.
	\end{itemize}
\end{itemize}

\section{To Do}

Currently, there is no need to add additional features to the backend API.

\chapter{Android applcation}

\section{Ready}

\begin{itemize}
	\item {\bf Page 1 : Login}. This page is fully working. The user can log in or register by clicking on the "Sign in" button.
	\item {\bf Page 2 : Sign in}. This page is fully working. The user can register by filling the form and pressing the button "Create an account".
	\item {\bf Page 3 : Home}. The map is displayed on the background. Also, we can search for a ride, or navigate trough the navbar.
	\item {\bf Page 4 : Navbar}. The navbar is fully working. The user can access to his profile, his lifts, page help, settings and logout.
	\item {\bf Page 5 : Lift}. This page is working. The search is working also. The user can select his origin, his destination, and the date where he want to travel.
	\item {\bf Page 6 : Availables drives}. This page is fully working. You can click on a available ride to display it in the page 7. You can also click on "I want to drive" to create your own ride.
	\item {\bf Page 7 : User x's route}. This page is fully working. You can see the driver's route, and also the path between your starting point and the meeting point with the driver. Idem for the your ending point and the dropping point.
	\item {\bf Page 8 : Create your route}. This page is to display the route that the user is going to create. He can edit this route, by clicking on the "Edit" button, to change a parameter. He will be redirected in page 5. Else, he can also click on the button "Create" to create his route.  He will be redirected to page 3.
	\item {\bf Page 9 : User profile}. This page is to display a specific user profile. It shows some informations, such as the username, the first name and last name, the phone number, the email and also the rating of the user.
	 \item {\bf Page 10 : My drives}. This page is to display the drives of the connected user. If he clicks on a drive, he will be redirected to page 8.
\end{itemize}

\section{To Do}
\begin{itemize}
\item {\bf Page 3 : Home}. The map is not focused on the user's location.
\item {\bf Page 5 : Lift}. There is no button to enter the current location of the user in the origin field. Also, the management of the weekly recurrence is not implemented.
\end{itemize}

\chapter{iOS application}

TODO