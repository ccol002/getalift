\chapter{iOS Application}

\section{How to install GetALift (GEA) on you iPhone}

\subsection{Xcode}

To work on the GAL project, you have to use Xcode. Xcode downloads directly on the App Store.
\\\\
To open the project on Xcode, launch Xcode and click on \textit{File/Open} and select in the  \textit{getalift-ios} folder the \textit{GALDev.xcodeproject} and click on \textit{Open}
\\\\
When you open it for the first time you should have errors concerning ???.

\subsection{???}


You should have an error concerning the \textit{NotificationBannerSwift}. It because of CocoaPods.

\subsection{CocoaPods}

CocoaPods is a dependency manager for Swift and Objective-C Cocoa projects. It has over 50 thousand libraries. It is used in the GAL application and it is necessary to install it is necessary to install it on your Mac.

\subsubsection{How to install CocoaPods on your Mac}

To install CocoaPods, you have to use the Terminal on your Mac. Open the Terminal.
\\\\
In the folder that contains the project : \textit{/getalift/getalift-ios/} make sure there is the file : \textit{Podfile}
\\\\
If it is not the case, in the Terminal, in the folder containing the xcodeproj file for your project, run the command:
\\\\
\begin{DDbox}{\linewidth}
\begin{verbatim}
pod init
\end{verbatim}
\end{DDbox}
\\\\
This will create a new text file named Podfile (no extension), with the following content:
\begin{verbatim}

\end{verbatim}
